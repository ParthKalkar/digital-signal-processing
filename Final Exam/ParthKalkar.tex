\documentclass[12pt,a4paper]{article}
\makeatletter
\renewcommand{\@evenhead}{}
\renewcommand{\@oddhead}{}
\renewcommand{\@evenfoot}{}
\renewcommand{\@oddfoot}{}
\makeatother
\usepackage{cmap}
\usepackage{graphicx}
\graphicspath{ {./} }
\usepackage{euscript}
\usepackage[T2A]{fontenc}
\usepackage[cp1251]{inputenc}
\usepackage[russian]{babel}
\usepackage{amssymb, amsmath}
\abovedisplayskip=.4\abovedisplayskip
\belowdisplayskip=.4\belowdisplayskip
\usepackage[left=1cm,right=1cm,top=1cm,bottom=19mm]{geometry}
\usepackage{tikz}
\usetikzlibrary{calc,intersections}

\begin{document}

\begin{center}
    \section*{Final examination\\
    Digital Signal Processing (S22)\\}
    Parth Kalkar BS19-DS-01\\
    Date of birth: 09.08.2000\\
    
\end{center}

\section{Task 1}
\subsection{Solution}

We start by substituting numbers into the defined operations and we get:

$$(v|w)=(-1)^{9}\cdot8\cdot v_1w_1^*+2000\cdot v_2w_2^*$$
$$(v|w)=-8\cdot v_1w_1^*+2000\cdot v_2w_2^*$$

From Lecture notes, Topic 1, Slide 32, we know that: To check if any operation is dot-product we need to consider axioms of inner product. These axioms for all vectors $u,v,w \in \pmb{C}$  any complex vector space and all scalars $a,b\in F$ any field of numbers are:
\begin{itemize}
    \item Conjugate symmetry: $(u|v) = (v|u)^*$, where $^*$ is a conjugate
operation on scalars.
    \item Linearity in the first argument: $(au+bv|w) = a(u|w)+b(v|w)$.
    \item Positive-definiteness: $(u|u) > 0$ assuming $u\neq 0$ and $(0|0) = 0$.
\end{itemize}


Now let us explain which dot-product axioms do hold and which do fail.:
\subsubsection{Conjugate symmetry}
    
    Rewriting $(w|v)^*$ by definition of the given operation:
    $$(w|v)^*=(-8\cdot w_1v_1^*+2000\cdot w_2v_2^*)^*$$
    
    Simplifying, using the properties of conjugates:
    $$(w|v)^*=(-8\cdot w_1v_1^*)^*+(2000\cdot w_2v_2^*)^*=$$
    $$=-8\cdot w_1^*(v_1^*)^*+2000\cdot w_2^*(v_2^*)^*=$$
    $$=-8\cdot w_1^*v_1+2000\cdot w_2^*v_2$$
    
    Applying commutative law for multiplication:
    $$(w|v)^*=-8\cdot v_1w_1^*+2000\cdot v_2w_2^*$$
    
  As a result,  $$(v|w) = (w|v)^* \Rightarrow \textbf{Conjugate symmetry axiom holds.}$$
    
\subsubsection{Linearity in the first argument}
    
    Taking scalars $a$ and $b\in F$ , vector $u\in \pmb{C}$ and expression $(av+bw|u)$:
    
    $$(av+bw|u) = -8(av_1+bw_1)u_1^*+2000(av_2+bw_2)u_2^*$$
    \newpage
    Applying distributive property of multiplication over addition: 
    $$(av+bw|u) = a(-8v_1u_1^*+2000v_2u_2^*)+b(-8w_1u_1^*+2000_2u_2^*)$$
    
   As a result, $$(av+bw|u) = a(v|u)+b(w|u)\Rightarrow \textbf{Linearity in the first argument axiom holds}$$
    
\subsubsection{Positive-definiteness} 
    
    Rewriting expression for $(v|v)$:
    $$(v|v) = -8v_1v_1^*+2000v_2v_2^*$$
    
    Since $v_1$ and $v_2$ are complex numbers, they can be written as: 
    $$v_1=x_1+iy_1,\ v_2=x_2+iy_2$$
    where $x_1,y_1,x_2,y_2\in \mathbb{R}$ and $i$ - iota: imaginary unit, which means 
    $$v_kv_k^*=(x_k+iy_k)(x_k-iy_k)=x_k^2+y_k^2=|v_k|^2$$
    where $|v_k|$ argument of complex number $v_k$, $k\in \mathbb{N}$.\\
    
    Using the above stated property:
    $$(v|v) = -8|v_1|^2+2000|v_2|^2$$

    The expression above is not always positive-definite because it contains negative number, that means it can be negative for some values of $v_1$ and $v_2$. For example, if $v_1=50$ and $v_2=1$ we get $$(v|v)=-20000+2000=-18000 < 0$$  \\ 
    As a result, \textbf{Positive-definiteness axiom fails}

\subsection{Conclusion}
The operations fails on one of the dot-product axioms, therefore the defined operation is \textbf{not a dot-product}.

\section{Task 3}

\subsection{Notation}
The day.month.year is given by $09.08.2000$, therefore $a = \begin{pmatrix} \boldsymbol{0} & 9 & 0 & 8 \end{pmatrix}$ and $b = \begin{pmatrix} \boldsymbol{2} & 0 & 0 & 0 \end{pmatrix}$.

\subsection{Solution}
    \subsubsection{Cross-Correlation}
    
        \begin{itemize}
        
            \item Definition:\\
        From Lecture Notes, Topic 3, slide 10: The deterministic cross-correlation of two signals $a$ and $b$ is a sequence $c_{a,b} = \begin{pmatrix}...c_{-3} & c_{-2} & c_{-1} & \boldsymbol{c_0} & c_1 & c_2 & c_3... \end{pmatrix}$, such that 
        $$c_n = \sum_{k=-\infty}^{k=\infty} a_kb^*_{k-n} = \langle a_k, b_{k-n}\rangle_k$$\\
        For all $q<0$, $q>3$ $a_q = 0$, $b_q = 0$, where $q \in \boldsymbol{Z}$, hence $c_n = \langle a_q, b_{q-n}\rangle_q = 0$\\
        \\
        In order to compute circular convolution we need to extend sequences to infinite sequences with \emph{finite support}, therefore we add zeros in the beginning and the end of sequences $a$ and $b$:
    $$a=(...\ 0\ 0\ \pmb{9}\ 0\ 8\ 0\ 0\ ...)$$$$b=(...\ 0\ 0\ \pmb{2}\ 0\ 0\ 0\ 0\ 0\ ...)$$
    

        \item Calculating $c_n$ for $-3\leq n \leq 3$, $n \in \boldsymbol{Z}$:
        
        $c_{-3} = a_0b_3 + ... = 0\cdot 0 + ... = 0$ \\
        $c_{-2} = a_0b_2 + a_1b_3 + ... = 0\cdot 0 + 9\cdot 0 + ... = 0$ \\
        $c_{-1} = a_0b_1 + a_1b_2 + a_2b_3 + ... = 0\cdot 0 + 9\cdot 0 + 0\cdot 0 + ... = 0$ \\
        $c_0 = a_0b_0 + a_1b_1 + a_2b_2 + a_3b_3 + ... = 0\cdot 2 + 9\cdot 0 + 0\cdot 0 + 8\cdot 0 + ... = 0$ \\
        $c_1 = a_0b_{-1} + a_1b_0 + a_2b_1 + a_3b_2 + ... = 0\cdot 0 + 9\cdot 2 + 0\cdot 0 + 8\cdot 0 + ... = 18$ \\
        $c_{2} = a_2b_0 + a_3b_1 + ... = 0\cdot 2 + 8\cdot 0 + ... = 0$ \\
        $c_{3} = a_3b_0 + ... = 8\cdot 2 + ... = 16$
        \end{itemize}
     
    \subsubsection{Convolution}
        \begin{itemize}
            \item Definition:\\
            From Lecture Notes, Topic 3, Slide 31: The convolution $(a * b)$ between signals $a$ and $b$ is a sequence, such that: $$ (a * b)_n = z_n = \sum_{k \in \boldsymbol{Z}} a_{n-k}b_k = \sum_{k \in \boldsymbol{Z}} a_kb_{n-k} $$
            
            For all $q<0$, $q>3$ $a_q = 0$, $b_q = 0$, where $q \in \boldsymbol{Z}$, hence $z_q = 0$ \\
            
             Just like last previous sub task, we will extend sequences $a$ and $b$ to infinite sequences with \emph{finite support} and add zeros at the start and end. \\
             
             It is important to note that the product of elements $a_i$ and $b_i$ where $i<-3$ and $i>3$ by any other element leads to 0, Therefore they can be excluded.
             
            \item Calculating $z_n$ for non-negative $n$, $n \in \boldsymbol{Z}$:
            
             $z_0 = \sum_{k \in \boldsymbol{Z}} a_{0-k}b_k =... + a_0b_0 + a_{-1}b_1 + a_{-2}b_2 + a_{-3}b_3 + ... = a_0b_0 = 0\cdot 2 = 0$ \\
             $z_1 = \sum_{k \in \boldsymbol{Z}} a_{1-k}b_k =... + a_1b_0 + a_{0}b_1 + ... = 9\cdot 2 + 0\cdot 0 = 18$ \\
             $z_2 = \sum_{k \in \boldsymbol{Z}} a_{2-k}b_k =... + a_2b_0 + a_{1}b_1 + a_{0}b_2 + ... = 0\cdot 2 + 9\cdot 0 + 0\cdot 0 = 0$ \\
             $z_3 = \sum_{k \in \boldsymbol{Z}} a_{3-k}b_k =... + a_3b_0 + a_{2}b_1 + a_{1}b_2 + a_0b_3 + ... = 8\cdot 2 + 0\cdot 0 + 9\cdot 0 + 0\cdot 0 = 16$
        \end{itemize}
        
        
    \subsubsection{Circular Convolution}
    
    \begin{itemize}
    
        \item Definition:\\
        From Lecture Notes, Topic 4, Slide 43: The circular convolution $(a^{(*)} b)$ between two finite (complex-valued) sequences $a$ and $b$ of some fixed length $m > 0$ is defined as the following sequence of the same length $m$:
        $$ (a^{(*)} b)_n = \sum_{k=0}^{k=m-1}a_kb_{(n-k)mod(m)}  = \sum_{k=0}^{k=m-1}a_{(n-k)mod(m)}b_k$$
        
        The length of the signal is 4, i.e. $m=4$, therefore formula of circular convolution for signals $a$ and $b$ is:
    $$(a^{(*)}b)_n=\sum^{k=3}_{k=0}a_{(n-k)\text{mod}4}b_k=\sum^{k=3}_{k=0}b_{(n-k)\text{mod}4}a_k$$
        
        \item Calculating $(a^{(*)} b)_n$ for $0 \leq n \leq 3$:
        
        $(a^{(*)} b)_3 = a_0b_3 + a_1b_2 + a_2b_1 + a_3b_0 = 0\cdot 0 + 9\cdot 0 + 0\cdot 0 + 8\cdot 2    = 16$ \\
        $(a^{(*)} b)_2 = a_0b_2 + a_1b_1 + a_2b_0 + a_3b_3 = 0\cdot 0 + 9\cdot 0 + 0\cdot 2 + 8\cdot 0 = 0$ \\
        $(a^{(*)} b)_1 = a_0b_1 + a_1b_0 + a_3b_2 + a_2b_3 = 0\cdot 0 + 9\cdot 2 + 8\cdot 0 + 0\cdot 0 = 18$ \\
        $(a^{(*)} b)_0 = a_1b_3 + a_2b_2 + a_3b_1 + a_0b_0 = 9\cdot 0 + 0\cdot 0 + 8\cdot 0 + 0\cdot 2 = 0$

        
    \end{itemize}\
    
    
    \subsection{Conclusion}
    \begin{enumerate}
        \item The cross-correlation of signals $a$ and $b$ is $c_{a,b} = \begin{pmatrix}...0 & 0 & 0 & \boldsymbol{0} & 18 & 0 & 16... \end{pmatrix}$
        
        \item The convolution of $(a * b)_0 = 0$, $(a * b)_1 = 18$, $(a * b)_2 = 0$, $(a * b)_3 = 16$
        
        \item The circular convolution is as follows: $(a^{(*)} b)_0 = 0$, $(a^{(*)} b)_1 = 18$, $(a^{(*)} b)_2 = 0$, $(a^{(*)} b)_3 = 16$.
     
    \end{enumerate}
     
     
    
\section{Task 4}
\subsection{Solution}

Substituting numbers into formula for output signal, we get:

$$(y_n)=\Big(\frac{8x_{n-1}+9x_{n+1}}{2000}\Big)$$

Exploring the properties of this system. 

\subsubsection{Linear}
    
    \begin{itemize}
        \item Definition: \\
        From Lecture Notes, Topic 3, Slide 13: A linear system enjoys additive and scaling that together are known in engineering \emph{as the superposition principle}: $T(ax+bu)=aT(x)+bT(u)$. \\

Let us consider equation for $T(ax+bu)$:
    $$T(ax+bu)=\frac{8(ax_{n-1}+bu_{n-1})+9(ax_{n+1}+bu_{n+1})}{2000}=$$$$=\frac{a(8x_{n-1}+9x_{n+1})}{2000}+\frac{b(8u_{n-1}+9u_{n+1})}{2000}=aT(x)+bT(u)$$
    Hence we can conclude that this system is \textbf{linear}.
    
    \item   Infinite matrix:
    $$\begin{pmatrix}
    ... & ... & ... & ... & ...\\
    ... & 0 & \frac{9}{2000} & 0 & ...\\
    ... & \frac{8}{2000} & 0 & \frac{9}{2000} & ...\\
    ... & 0 & \frac{8}{2000} & 0 & ...\\
    ... & ... & ... & ... & ...
    \end{pmatrix}$$
    
    \end{itemize}
 
 \subsubsection{Memoryless}
    
    \begin{itemize}
        \item Definition: \\
        From Lecture notes, Topic 3, Slide 15: A system $T$ is \emph{memoryless} if for every $k\in\mathbb{Z}$ and all input signals $x$ and $x'$ the following implication holds: $x_k=x'_k$ implies $(Tx)_k=(Tx')_k$.
    
        Writing equations for $(Tx)_k$ and $(Tx')_k$:
    $$(Tx)_k=\Big(\frac{8x_{k-1}+9x_{k+1}}{2000}\Big),\ (Tx')_k=\Big(\frac{8x'_{k-1}+9x'_{k+1}}{2000}\Big)$$
    
    The formulas for $(Tx)_k$ and $(Tx')_k$ contain elements $x_{k-1}$, $x'_{k-1}$, $x_{k+1}$ and $x'_{k+1}$ . We do not have information if they are equal to each other, therefore we cannot surely say that this system is memory-less. Hence, the above mentioned system is \textbf{not memory-less}.
    \end{itemize}

\subsubsection{Causal}
    
    \begin{itemize}
        \item Definition: \\
        From Lecture notes, Topic 3, Slide 15: A system $T$ is called \emph{causal} if for every $k\in \mathbb{Z}$ and all input signals $x$ and $x'$ , the following implication holds: $x_{(-\infty,k]}=x'_{(-\infty,k]}$ implies $(Tx)x_{(-\infty,k]}=(Tx')x_{(-\infty,k]}$.
        \\
    
    Similar to the previous sub task(memory-less), equations for $(Tx)_k$ and $(Tx')_k$ contain elements $x_{k+1}$ and $x'_{k+1}$ that do not belong to $x_{(-\infty,k]}$ and $x'_{(-\infty,k]}$ respectively, therefore we cannot surely say that these elements are equal. Hence the above mentioned system is \textbf{not causal}
    \end{itemize}

  \subsubsection{Shift-invariant}
    \begin{itemize}
        \item Definition:\\
    From Lecture notes, Topic 3, Slide 16: A linear system $T$ is \emph{shift-invariant} (LSI) if for every input signal $T$ (shifted $x$) is shifted $Tx$, i.t. if $\big((Tx)_{n-k}\big)=T(x_{n-k})$ for every input signal $x$ and $k\in\mathbb{Z}$.
    \\
    
    Let us consider equations for $\big((Tx)_{n-k}\big)$ and $T(x_{n-k})$:
    $$\big((Tx)_{n-k}\big)=(y)_{n-k}=\frac{8x_{n-k-1}+9x_{n-k+1}}{2000}$$
    $$T(x_{n-k})=\frac{8x_{n-k-1}+9x_{n-k+1}}{2000}$$
    
    As a result, it is clear that these equations are equal to each other, therefore $\big((Tx)_{n-k}\big)=T(x_{n-k})$. Hence the above mentioned system is \textbf{shift-invariant}
    \end{itemize}
    
    
    
  
    
    \subsubsection{BIBO-stable}
    \begin{itemize}
        \item  Definition: \\ 
    From Lecture notes, Topic 3, Slide 18: A system $T$ is called \emph{ bounded-input, bounded-output stable} (BIBO-stable) if a bounded input always produces bounded output: $Tx\in l^{\infty}$ fir all $x\in l^{\infty}$.
    \\
    
     From the same lecture notes, according to the theorem, the LSI system is BIBO-stable iff its impulse response is absolutely summable. The impulse response of this system is:
    $$(..., 0, 0, \frac{9}{2000}, 0, \frac{8}{2000}, 0, 0, ...)$$
    Therefore the sum of the response can be computed:
    $$\frac{9}{2000}+\frac{8}{2000}=\frac{17}{2000}$$
    
    As a result, it is clear that the impulse response is summable. Hence the above mentioned system is \textbf{BIBO-stable}.

    \end{itemize}
   
    \subsection{Conclusion}
    The given system is linear, shift-invariant and BIBO-stable, however, it is not memory-less and causal.

    
 \section{Task 5}
 
 \subsection{Notation}
For the calculations below, the signal $a = \begin{pmatrix} 0 & 9 & 0 & 8 \end{pmatrix}$ 

 \subsection{Solution}
\subsubsection{DTFT}
 \begin{itemize}  
 \item Definition:\\
    From Lecture notes, Topic 4, Slide 9, The Discrete-Time Fourier Transform maps each filter $a$ to the frequency response spectrum function
    $$A(e^{j\omega }) = \sum_{k \in \boldsymbol{Z}} e^{-j\omega k} a_k$$
    of real argument $\omega$
     \item Computing DTFT $A(e^{j\omega}):$
  
        $$ A(e^{j\omega}) = \sum_{k \in \boldsymbol{Z}} e^{-j\omega k} a_k = ... + a_0e^{-0j\omega} + a_1e^{-j\omega} + a_2e^{-2j\omega} + a_3e^{-3j\omega} + ... $$
        
        $$ A(e^{j\omega}) = ... + 0\cdot 1 + 9e^{-j\omega} + 0e^{-2j\omega} + 8e^{-3j\omega} + ... $$
        
        For all $k<0$ or $k>3$ $a_k = 0$, therefore their sum is zero. Hence:
        
        $$ A(e^{j\omega}) = 9e^{-j\omega} + 8e^{-3j\omega}  $$
\end{itemize}

\subsubsection{IDTFT}  
\begin{itemize}
    \item Definition:\\
    From Lecture Notes, Topic 4, Slide 10: The Inverse-DTFT of a $2\pi$-periodic function $A(e^{j\omega})$ of the real argument is the following (two-side infinite) sequence $a = \begin{pmatrix} ...a_{-3} & a_{-2} & a_{-1} & \boldsymbol{a_0} & a_1 & a_2 & a_3 \end{pmatrix}$, where 
    $$ a_n = \frac{1}{2\pi} \int_{-\pi}^{\pi} A(e^{j\omega})e^{j\omega n}d\omega $$
    for all $n \in \boldsymbol Z$
    Find $a_n$ for $-3\leq n \leq 3$:
  
    Let us start by calculating $a_{-3}$:
        $$ a_{-3} =  \frac{1}{2\pi} \int_{-\pi}^{\pi} (9e^{-j\omega} + 8e^{-3j\omega})e^{-3j\omega}d\omega =$$
        $$ = \frac{1}{2\pi} \int_{-\pi}^{\pi} (9e^{-4j\omega} + 8e^{-6j\omega})d\omega = \frac{9}{8\pi}(e^{-4j\pi} - e^{4j\pi}) + \frac{2}{3\pi}(e^{-6j\pi} - e^{6j\pi}) =  $$
        $$ = \frac{9}{8\pi}(cos(-4\pi) + jsin(-4\pi)) - \frac{9}{8\pi}\cdot 1 + \frac{2}{3\pi}(cos(-6\pi) + jsin(-6\pi)) - \frac{2}{3\pi}\cdot 1 = $$
        
        $$ a_{-3} = \frac{9}{8\pi} - \frac{9}{8\pi} + \frac{2}{3\pi} - \frac{2}{3\pi} = 0 $$

         Similarly we can calculate $a_{-2}$, $a_{-1}$, and $a_0$ as follows:
         
        $$ a_{-2} = - \frac{3}{2\pi} + \frac{3}{2\pi} - \frac{4}{5\pi} + \frac{4}{5\pi} = 0 $$
    
        $$ a_{-1} = \frac{9}{4\pi} - \frac{9}{4\pi} + \frac{1}{\pi} - \frac{1}{\pi} = 0$$
        
        $$a_0 = - \frac{1}{2\pi} + \frac{1}{2\pi} - \frac{4}{3\pi} + \frac{4}{3\pi} = 0$$
    
        
        Now we can calculate $a_1$ as follows:
        $$ a_1 =  \frac{1}{2\pi} \int_{-\pi}^{\pi} (9e^{-j\omega} + 8e^{-3j\omega})e^{j\omega}d\omega = \frac{1}{2\pi} \int_{-\pi}^{\pi} (9 + 8e^{-2j\omega})d\omega =$$
        $$ = \frac{9}{2\pi}(\pi - (-\pi)) + \frac{2}{\pi}(e^{-2j\pi} - e^{2j\pi}) =  $$
        $$ = 9 + \frac{2}{\pi}(cos(-2\pi) + jsin(-2\pi)) - \frac{2}{\pi}\cdot 1 = $$
        
         $$a_1 = 9 + \frac{2}{\pi} - \frac{2}{\pi} = 9$$
         
      
        
        Similarly calculating $a_2$ as follows:
        
        $$ a_2 = - \frac{9}{2\pi} + \frac{9}{2\pi} - \frac{4}{\pi} + \frac{4}{\pi} = 0$$
        
        Finally calculating $a_3$ as follows:
        $$ a_3 =  \frac{1}{2\pi} \int_{-\pi}^{\pi} (9e^{-j\omega} + 8e^{-3j\omega})e^{3j\omega}d\omega = \frac{1}{2\pi} \int_{-\pi}^{\pi} (9e^{2j\omega} + 8)d\omega =$$
        $$ =\frac{9}{4\pi}(e^{2j\pi} - e^{-2j\pi}) + \frac{4}{\pi}(\pi +\pi) =$$
        $$ = \frac{9}{4\pi}(cos(2\pi) + jsin(2\pi)) - \frac{9}{4\pi}(cos(-2\pi) + jsin(-2\pi)) + 8 = $$
        
        $$  a_3 = \frac{9}{4\pi} - \frac{9}{4\pi} + 8 = 8 $$

    As a result, $a = \begin{pmatrix} ...a_{-3} & a_{-2} & a_{-1} & \boldsymbol{a_0} & a_1 & a_2 & a_3... \end{pmatrix} = \begin{pmatrix} ...0 & 0 & 0 & \boldsymbol{0} & 9 & 0 & 8... \end{pmatrix}$, that is exactly the same as the given signal $a = \begin{pmatrix} \boldsymbol{0} & 9 & 0 & 8 \end{pmatrix}$\\

\end{itemize}
\subsection{Conclusion}
After computing DTFT and IDTFT we can validate that in this case IDTFT is the inverse for DTFT indeed. Therefore, it is clear that IDTFT is indeed the inverse of $A(e^{j\omega})$.


\end{document}